\documentclass[conference]{IEEEtran}

\usepackage[dutch]{babel}

\begin{document}

\title{Project blackjack}

\author{
\IEEEauthorblockN{Seyed Kavoos Boloorchi}
\IEEEauthorblockA{
Hogeschool Gent\\
Bedrijf en organisatie\\
Email: seyedkavoos.boloorchi.u1660@student.hogent.be}
\and
\IEEEauthorblockN{Daan Flor\'e}
\IEEEauthorblockA{
Hogeschool Gent\\
Bedrijf en organisatie\\
Email: daan.flore.u8852@student.hogent.be}
\and
\IEEEauthorblockN{Robin Lodrigo}
\IEEEauthorblockA{
Hogeschool Gent\\
Bedrijf en organisatie\\
Email: robin.lodrigo.u8341@student.hogent.be}
\and
\IEEEauthorblockN{Jonas Wallays}
\IEEEauthorblockA{
Hogeschool Gent\\
Bedrijf en organisatie\\
Email: jonas.wallays.u8978@student.hogent.be}}


\maketitle

\begin{abstract}
The abstract goes here.
\end{abstract}

\IEEEpeerreviewmaketitle

\section{Introductie}
Voor het kaartspel blackjack gaan wij de 2 meest bekende strategieën namelijk basic en perfect strategie vergelijken en kijken hoe we de basic kunnen aanpassen om de winst te verhogen.

 
\hfill April 20, 2015

\section{Welke strategie heeft de meeste kans op winst?}

\subsection{Wat zijn de gebruikte strategieën}
De 2 strategieën die wij gaan gebruiken zijn de basic en perfect strategy pattern. We hebben deze gekozen omdat die het meest gekend zijn.

\subsection{Hoe worden de strategieën in de praktijk toegepast?}

\subsection{Hoe evolueerden de strategieën doorheen de tijd?}

\subsection{Hoe kunnen we de strategieën verbeteren?}
\subsubsection{Basic}
Na het onderzoeken van de strategie hebben we kunnen concluderen dat de strategie geoptimaliseerd kan worden om meer winst te maken.
We hebben onze aangepaste tabel in een simulator gestoken en kunnen zien dat we met onze aanpassingen meer winst konden maken op langere termijn.

hier komen 2 afbeelding oude tabel nieuwe en dan resultaten
\subsubsection{Perfect}

\subsection{ Waar zal voor elke strategie de kans op winst stagneren, waar ligt het optimale punt van die bepaalde strategie?}
Voor bijde strategieën is het wel duidelijk dat de kans op winst vergroot als het aantal gespeelde rondes verhoogd, wat we zien is dat bij 100 rondes er nog altijd verlies kan gemaakt worden omdat het ook nog altijd afhangt van de kaarten die de speler krijgt maar als we dan meer rondes beginnen te spelen zoals 1000 kunnen we wel zien dat er winst begint te komen Hierdoor kunnen we zeggen dat de resultaten wel lichtjes beïnvloedbaar zijn maar dat men daar een groot aantal rondes voor moet spelen eer het resultaat zichtbaar wordt. 
\section{Conclusion}
The conclusion goes here.

\section*{Acknowledgment}

The authors would like to thank...

\begin{thebibliography}{1}

\bibitem{IEEEhowto:kopka}
H.~Kopka and P.~W. Daly, \emph{A Guide to \LaTeX}, 3rd~ed.\hskip 1em plus
  0.5em minus 0.4em\relax Harlow, England: Addison-Wesley, 1999.

\end{thebibliography}

\end{document}


